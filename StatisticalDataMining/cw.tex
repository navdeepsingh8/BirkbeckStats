\documentclass[a4paper,11pt]{article}

\usepackage{fancyhdr}
\usepackage[left=2.5cm,top=3.5cm,headheight=2cm,right=3cm,bottom=3cm]{geometry}
\usepackage{bm}
\usepackage{amsmath,amssymb}

\pagestyle{fancyplain}
\author{Navdeep Singh Daheley}
\title{Statistical Data Mining\\Coursework}
\lhead[\thepage]{Navdeep Singh Daheley\\MSc Applied Statistics}
\rhead[\thepage]{Statistical Data Mining Coursework - \today}
\setlength{\parskip}{0.3cm}
\setlength{\parindent}{0cm}



\usepackage{Sweave}
\begin{document}
\setkeys{Gin}{width=0.6\textwidth}
\maketitle
\tableofcontents

\newpage
\section{Initial exploration, summary statistics, plots}

Approach of not looking at test data during model building (free from
data snooping bias).

Steps
\begin{enumerate}
  \item Unconditional class probabilities in the training data
  \item Summary statistics of the explanatory variables
  \item Univariate densities: with and without class labels, and
    comments about distributional shapes, outliers and separation
  \item Scatter plot matrices: with and without class labels and
    comments about separation
  \item Correlation matrix tables (Pearson and Spearman) to illustrate
    numerical linear dependency measures
  \item PCA with/without outliers: scree plot and scatter plot matrix in PC-space
  \item Classical scaling using Spearman rank correlations to account
    for non-linearities (ordinal scaling?)
  \item (if have time) Sammon mapping as another non-linear dimensional scaling procedure
\end{enumerate}

Consider talking about Bayes risk here - definition, and what it is
for the training set. A note about generalization to the test set.

\newpage
\section{Linear and quadratic discriminant analysis}

Note the model equation and main features (from notes, and look up
link with canonical variates analysis).\\


\end{document}

