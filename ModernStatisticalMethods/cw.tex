\documentclass[a4paper,11pt]{article}
\usepackage{fancyhdr}
\usepackage[left=2.5cm,top=3.5cm,headheight=2cm,right=3cm,bottom=3cm]{geometry}
\usepackage{bm}
\usepackage{amsmath,amssymb}
\pagestyle{fancyplain}
\lhead[\thepage]{Navdeep Daheley\\MSc Applied Statistics}
\rhead[\thepage]{Modern Statistical Methods - Coursework - \today}
\setlength{\parskip}{0.3cm}
\setlength{\parindent}{0cm}



\usepackage{Sweave}
\begin{document}
\setkeys{Gin}{width=0.6\textwidth}
{\bf Please note that some of the referenced figures are at the back of the document.}

1. (a) The cumulative distribution function (cdf) $F(x)$ corresponding to the
Weibull random variable $X$ parameterised by $\rho > 0$ and $\kappa > 0$ is
\begin{equation*}
  F(x) = 1 - e^{-(\rho x)^\kappa}
\end{equation*}
for $x \ge 0$. Since $F(X)$ is uniformly distributed on $[0,1]$ and it
is an increasing function on this interval, we have the result that
for any $U \sim U(0,1)$, $F^{-1}(U)$ has cdf $F$. This result is the
basis for simulation of $X$ via the  inversion method. It is
sufficient to evaluate $F^{-1}(u)$:
\begin{eqnarray*}
  u & = & 1 - e^{-(\rho x)^\kappa} \\
  e^{-(\rho x)^\kappa} & = & 1-u\\
%  e^{-(\rho x)^\kappa} & = & U\\
  (\rho x)^\kappa & = & -\ln(1-u) \\
  x & = & \frac{1}{\rho} \{ -ln(1-u) \} ^{\frac{1}{\kappa}}\\
  \end{eqnarray*}
If $u\sim U(0,1)$ then so is $1-u$. To get the result we replace this with $U$ and $x$ by
$X$ which is a random variable with cdf $F(x)$:
\begin{equation*}
  X = \frac{1}{\rho} \{ -ln(U) \} ^{\frac{1}{\kappa}}
  \end{equation*}

1. (b) A brief summary of the theory behind the acceptance-rejection
(A-R) method:\\
Assume $f(x)$ and $h(x)$ are PDFs such that (the envelope)
$g(x)=kh(x)\ge f(x)$ $\forall x$, $k$ is typically the smallest
positive constant which makes this true
and $h(x)$ has support $[c,d]$. Then for random variables
$X$ and $Y$ if  $Y|(X=x) \sim U(0,g(x))$, $X|(Y\le f(X))$ has PDF
$f(x)$. In the A-R method, $f(x)$ is the PDF we wish to simulate from
and $h(x)$  is a PDF we can simulate from directly.\\
Here $f$ is given by
\begin{equation*}
  f(x) = \frac{1+\alpha}{\alpha(1+x)^2}
\end{equation*}
for $0\le x\le\alpha$ and $\alpha > 0$ and we are instructed to use
the uniform distribution to construct the envelope, i.e.
\begin{equation*}
  h(x) = \frac{1}{\alpha}
\end{equation*}
for $0\le x\le\alpha$. Furthermore we set $k$ to be the smallest
number that ensures $g(x)\ge f(x)$ for $0\le x\le\alpha$. Since $f(x)$
is strictly decreasing in $x$ its maximum value is attained at $x=0$ where
$f(0)=\alpha^{-1}(1+\alpha)$. Therefore we set $k=(1+\alpha)$ giving
the envelope
\begin{equation*}
  g(x) = \frac{1+\alpha}{\alpha}
\end{equation*}
The outline for simulating a sequence $\{x_i\}$ of independent observations from
$f(x)$ is as follows:
\begin{enumerate}
  \item Simulate a candidate observation $x_i \sim U(0,\alpha)$
    i.e. from $h(x)$
  \item Simulate $y_i \sim U(0,\alpha^{-1}(1+\alpha))$ i.e. $U(0,g(x_i))$

  \item If $y_i\le f(x_i)$ accept $x_i$ otherwise reject it
  \item Repeat the steps above
\end{enumerate}

The number of pseudorandom numbers generated per simulated observation
can be evaluated by considering the probability $p$ of accepting a
candidate observation:
\begin{eqnarray*}
 p & = &  \frac{\int_0^\alpha \! f(x) \, dx}{\int_0^\alpha \! g(x) \, dx}\\
   & = &  \frac{\alpha^{-1}(1+\alpha) \int_0^\alpha \! (1+x)^{-2} \, dx}
               {\alpha^{-1}(1+\alpha)\int_0^\alpha \!  \, dx} \\
   & = &  \frac{1}{\alpha} \int_0^\alpha \! (1+x)^{-2} \, dx\\
   & = &  \frac{1}{\alpha}\left[-(1+\alpha)^{-1} \right]^\alpha_0 \\
   & = &  \frac{1}{1+\alpha}
\end{eqnarray*}
Therefore the expected number of candidate observations $x_i$ generated per
accepted observation is $(1+\alpha)$, and since for each $x_i$ a
$y_i$ is also generated, on average $2(1+\alpha)$ pseudorandom numbers
are generated per observation simulated. Clearly, as $\alpha \rightarrow
0$, $f(x)$ tends to the envelope in area, reducing the number of
pseudorandom numbers required to simulate from it.

\begin{figure}[h]
\begin{center}
\includegraphics{cw-002}
\caption{$f(x)$ (curved) and $g(x)$ (horizontal) for $\alpha=$ 0.2,0.5
and 0.8}
\end{center}
\end{figure}


\pagebreak
2. (a) The data were input into R to obtain the data frame \texttt{dugong}, and
plotted.
\begin{Schunk}
\begin{Sinput}
> dugong=read.table(file(paste(path,"q2.txt",sep="")),header=T);
> plot(dugong$age,dugong$length);
\end{Sinput}
\end{Schunk}
\begin{figure}[h]
  \begin{center}
\includegraphics{cw-004}
\end{center}
\end{figure}
The model is a nonlinear regression model because of the presence of
both $\beta$ and $\gamma$ in the $\beta\gamma^{x_i}$ term. For
strictly non-negative $x_i$ as is the case here, $\mu_i$ is
bounded above by $\alpha$. This aspect of the model is consistent with
the length data, which are bounded above at around 3 metres. Maximum adult dugong
length is an interpretation of $\alpha$. Also, according to the model
$\mu_i$ increases with $x_i$ and is concave, a feature also exhibited in the
data. The extent of the concavity depends on $\gamma$, which can
therefore  be interpreted as a measure of how sensitive growth rate is
to age. $\gamma$ close to 0 corresponds to a much higher growth rate
in youth than in old age. $\gamma$ close to 1 corresponds to a more
stable growth rate through life. Finally $\alpha-\beta$ is the
intercept of the model, the value of $\mu_i$ when $x_i=0$. This can be
interpreted as the size of a newborn dugong, and given $\alpha$ is
governed by $\beta$.

2. (b)
\begin{Schunk}
\begin{Sinput}
> dugong.fit <- nls(length ~ alpha - beta * gamma^age, dugong,
+                   start=list(alpha=2.7,beta=1.0,gamma=0.8));
> dugong.fit;
\end{Sinput}
\begin{Soutput}
Nonlinear regression model
  model:  length ~ alpha - beta * gamma^age 
   data:  dugong 
 alpha   beta  gamma 
2.6581 0.9635 0.8715 
 residual sum-of-squares: 0.2177

Number of iterations to convergence: 6 
Achieved convergence tolerance: 3.291e-06 
\end{Soutput}
\begin{Sinput}
> dugong.fit1 <- nls(length ~ alpha - beta * gamma^age, dugong,
+                    start=list(alpha=2.5,beta=0.8,gamma=0.6));
> dugong.fit1;
\end{Sinput}
\begin{Soutput}
Nonlinear regression model
  model:  length ~ alpha - beta * gamma^age 
   data:  dugong 
 alpha   beta  gamma 
2.6581 0.9635 0.8715 
 residual sum-of-squares: 0.2177

Number of iterations to convergence: 7 
Achieved convergence tolerance: 9.317e-06 
\end{Soutput}
\begin{Sinput}
> plot(dugong$age, dugong$length);
> lines(dugong$age, predict(dugong.fit1));
\end{Sinput}
\end{Schunk}
\begin{figure}[h]
  \begin{center}
\includegraphics{cw-006}
\end{center}
\end{figure}
The output of the code shows the nonlinear least-square estimates of
$\alpha$,$\beta$ and $\gamma$ starting from two different sets of
starting values. The starting values appear not to be too critical as the
same estimates are arrived at in each case. The plot shows the fitted
values $\hat\mu_i$ against $x_i$, shown as a line (with linear
interpolation between data points). The sum of squared residuals
$(y_i-\hat\mu_i)^2$ has been minimised (via the Gauss-Newton numerical
method) by these parameter estimates. The model appears to
fit well with no (visually) discernable pattern in the
residuals. Further analysis of the residuals could be carried out such
as plots and tests of randomness. We also calculate the ratio of
the model sum-of-squares to the total-sum-of-squares.
\begin{Schunk}
\begin{Sinput}
> total_sum_sq <- (nrow(dugong)-1)*var(dugong$length);
> 1-deviance(dugong.fit1)/total_sum_sq;
\end{Sinput}
\begin{Soutput}
[1] 0.8889847
\end{Soutput}
\end{Schunk}
$89\%$ of the variance in dugong length is accounted for by the model
which is a good fit.

($\hat\alpha=$) 2.66 metres is the estimated upper bound for dugong
length. ($\hat\alpha$-$\hat\beta$=) 1.69 metres is the estimated
lower bound for dugong length. Also $\hat\gamma=$0.87 corresponds to a
growth rate that is highest before the age of 10, subsequently falling to a
visually lower rate. Estimates of growth rates could be obtained by
linearising parts of the fitted response function and calculating its
gradient.

To carry out classical inference about the precision of these parameter
estimates we would need to make distributional assumptions about the
residuals. Furthermore since the model is nonlinear, we cannot apply
the machinery associated with OLS estimators here.

2. (c) The following modified code was run in R, making use of the
\texttt{boot} package.
\begin{Schunk}
\begin{Sinput}
> library(boot)
> set.seed(965) #Starting at seed 964 causes nls() in R to not converge
> #A function which estimates the coefficients using nls() from data d
> dugong.fit <- function(d) {
+     coef(nls(length ~ alpha - beta * gamma^age, d,
+           start=list(alpha=2.5,beta=0.8,gamma=0.6)));
+ }
> #A wrapper function which calls the estimation function using data d,
> #bootstrapped according to vector of indices i
> row.fun <- function(d,i) {
+  dugong.fit(d[i,]);
+ }
> #A call to boot() which carries out bootstrap estimation with 1000 resamples
> dugong.boot <- boot(dugong,row.fun,R=1000,stype="i");
\end{Sinput}
\end{Schunk}

\begin{Schunk}
\begin{Soutput}
ORDINARY NONPARAMETRIC BOOTSTRAP


Call:
boot(data = dugong, statistic = row.fun, R = 1000, stype = "i")


Bootstrap Statistics :
     original       bias    std. error
t1* 2.6580766  0.010357532  0.08096630
t2* 0.9635221  0.013616427  0.06619887
t3* 0.8714584 -0.003533535  0.03511136
\end{Soutput}
\end{Schunk}

\begin{figure}[h]
  \begin{center}
\includegraphics{cw-010}
\includegraphics{cw-011}
\includegraphics{cw-012}
\caption{The output of the \texttt{plot} function applied to the
  \texttt{dugong.boot} object for each of $\alpha$, $\beta$ and
  $\gamma$ respectively. For each parameter there is a histogram
  of the bootstrap distribution and a $QQ$-plot vs quantiles of the
  standard normal.}
\label{fig:plotboot}
\end{center}
\end{figure}

Empirical bootstrap distributions for the model parameters $\alpha$, $\beta$ and
$\gamma$ (\texttt{t1*},\texttt{t2*} and \texttt{t3*} in the output)
have been computed in this analysis. The idea behind this is
that for a random sample from a population, in the absence of any
other knowledge of that population, the distribution in the sample is
the best guide to the distribution in the population. The population
distribution is approximated by resampling (with replacement) from the
sample. More specifically, the non-linear least squares estimates for
$\alpha$, $\beta$ and $\gamma$ have been calculated for 1000 samples
of the bivariate (age,length) data. Each sample (of equal size to the
original data set) has been obtained by sampling with replacement from
the original data set. The bootstrap distribution for each parameter,
an approximation to the population distribution, is based on these
1000 resampled estimates.

The bias and standard error of each of the parameter estimates have been
estimated from the bootstrap resamples, the bias as follows, using the
bias of $\hat\alpha$, $b(\alpha)=E[\hat\alpha]-\alpha$, as an example:
\begin{equation*}
  \hat{b}(\alpha)= \frac{1}{1000}\sum_{i=1}^{1000}\hat\alpha_i' - \hat\alpha  \\
\end{equation*}
where $\hat\alpha_i'$ is the estimate of $\alpha$ from the $i$th bootstrap
resample and $\hat\alpha$ is the estimate from the original
sample. The rationale for this is that the distribution of
$(\hat\alpha_i'-\hat\alpha)$ mimics that of
$\hat\alpha-\alpha$. The estimated bias for $\hat\alpha$ and
$\hat\beta$, $\hat{b}(\alpha)$ and $\hat{b}(\beta)$ are
positive suggesting $\alpha$ and $\beta$ may be slightly overestimated
by $\hat\alpha$ and $\hat\beta$. The amounts in each case are small:
the bias represents about 1.0 and 1.4cm
respectively. $\hat{b}(\gamma)$ is in relation very small and negative.

Applying the \texttt{plot} function to the \texttt{dugong.boot} object
for each parameter gave the output in Fig \ref{fig:plotboot}. For each
parameter there is a histogram of the bootstrap distribution and a
$QQ$-plot vs quantiles of the standard normal. The distribution of
$\hat\beta'$ is closest to normal whilst $\hat\alpha'$
is positively skewed and $\hat\gamma'$ negatively skewed.

Further R code was run to calculate the Percentile and Accelerated
Bias-Corrected Percentile (BCa) confidence intervals for the three
parameters. The \texttt{boot.ci} function was used to calculate these
confidence intervals and the output rounded to two decimal places.
\begin{Schunk}
\begin{Sinput}
> #A function which outputs Percentile and BCa confidence intervals for a
> #specified parameter from the model
> show.dugong.ci <- function(param.num) {
+     ci.param <- boot.ci(dugong.boot,conf=0.95,type=c("perc","bca"),index=param.num);
+     ci.out <- round(cbind(ci.param$perc[4:5],ci.param$bca[4:5]),2);
+     colnames(ci.out) <- c("Percentile","BCa");
+     rownames(ci.out) <- c("L","U");
+     ci.out;
+ }
> #Calculate confidence intervals for each parameter
> show.dugong.ci(1); #alpha
\end{Sinput}
\begin{Soutput}
  Percentile  BCa
L       2.52 2.52
U       2.85 2.84
\end{Soutput}
\begin{Sinput}
> show.dugong.ci(2); #beta
\end{Sinput}
\begin{Soutput}
  Percentile  BCa
L       0.85 0.80
U       1.11 1.08
\end{Soutput}
\begin{Sinput}
> show.dugong.ci(3); #gamma
\end{Sinput}
\begin{Soutput}
  Percentile  BCa
L       0.79 0.78
U       0.92 0.92
\end{Soutput}
\end{Schunk}

The BCa intervals are corrected for bias and skewness in the parameter
estimates. The assumption underlying these intervals is that for some
(unknown) monotonic increasing function $f$,
\begin{equation*}
  f(\hat\alpha) \sim N(f(\alpha)-z_0(1+af(\alpha)), 1+af(\alpha))
\end{equation*}
$z_0$ corresponds to the bias of $\hat\alpha$ and $a$ corresponds to
its skewness. The Percentile and BCa confidence intervals for
$\alpha$ and $\gamma$ are similar. For $\beta$ however the BCa
interval is (0.8,1.08) compared to the Percentile interval of
(0.85,1.11). This negative shift and stretch is probably due to the
relatively larger positive estimated bias in $\hat\beta$ of 0.014.

In conclusion, the 95\% confidence intervals for the three parameters
are reasonably well behaved (and in summary about 0.3 units wide for
$\alpha$ and $\beta$, 0.1 for $\gamma$), and in further analysis a
similar confidence interval could be constructed for average length,
$\mu_i$. This gives us some indication about the precision of the
parameter estimates. Furthermore we have estimated levels of bias,
which could be interpreted as being not significantly large.

\newpage
3. (a)
\begin{Schunk}
\begin{Sinput}
> library(MASS)
> cholA <- c(122, 233, 239, 291, 254, 312, 276, 250, 234, 246, 181, 197,
+            248, 268, 252, 224, 202, 329, 218, 325);
> cholB <- c(252, 420, 226, 185, 175, 263, 242, 246, 153, 224, 183, 212,
+            137, 188, 202, 250, 194, 148, 213, 169);
> summary(cholA);
\end{Sinput}
\begin{Soutput}
   Min. 1st Qu.  Median    Mean 3rd Qu.    Max. 
  122.0   222.5   247.0   245.1   270.0   329.0 
\end{Soutput}
\begin{Sinput}
> summary(cholB);
\end{Sinput}
\begin{Soutput}
   Min. 1st Qu.  Median    Mean 3rd Qu.    Max. 
  137.0   181.0   207.0   214.1   243.0   420.0 
\end{Soutput}
\begin{Sinput}
> chol.box <- boxplot(cholA,cholB,ylab="Cholesterol level",names=c("Type A","Type B"));
> chol.box$stats[3,]
\end{Sinput}
\begin{Soutput}
[1] 247 207
\end{Soutput}
\end{Schunk}
\includegraphics{cw-014}

A box plot of cholesterol levels has been produced for the two samples
of men corresponding to type A and type B. It appears from this plot
that type A men tend to have higher cholesterol levels than type
B. The quartiles, whiskers and the median are all higher for type A
than for type B men. The two medians are 247 and 207
respectively. The type A sample has a larger range but the type B
sample appears to have a larger inter-quartile range (IQR). There is an
outlier in each sample, as defined by observations at least 1.5 times
the IQR outside a quartile. For the type A sample this is a low
cholesterol level (below 160) whereas for type B it is a high level
(above 410).

3. (b) A kernel density estimate $\hat{f}_h$ of a pdf $f$ is given by
\begin{equation*}
 \hat{f}_h(x) = \frac{1}{n}\sum_{i=1}^n\frac{1}{h}K\left(\frac{x-x_i}{h}\right)
  \end{equation*}
where $K$ is a smoothing kernel, and $h$ the smoothing window
parameter or bandwidth. $K$ is usually chosen to be a pdf itself, and
it turns out the exact choice of functional form does not affect
$\hat{f}_h$ very much. A sensible choice is to use the standard
Gaussian pdf. $h$ is typically chosen to minimise the mean integrated
square error (MISE) criterion:
\begin{eqnarray*}
  \mbox{MISE} & = &
   E\left[\int_{-\infty}^{\infty}\! [\hat{f}_h(x)-f(x)]^2 \, dx \right]\\
              & = &
   \int_{-\infty}^{\infty}\! \mbox{Var}\left(\hat{f}_h(x)\right)\, dx +
   \int_{-\infty}^{\infty}\! \mbox{Bias}^2(x)\, dx\\
\end{eqnarray*}
Smaller $h$ reduces Bias$(x)$ whilst large $nh$ reduces
Var$(\hat{f}_h(x))$. Using Taylor approximations to $f$, it can be
shown that for a given sample size $n$, the more variable $f$ is, the
smaller the optimal value of $h$ is. Furthermore, assuming that the
smoothness of $f$ can be approximated by a Gaussian pdf
$N(\mu,\sigma^2)$, it can be shown that the MISE is minimised by
\begin{equation*}
  h_{opt} = k\sigma n^{-1/5}
\end{equation*}
where $k$ is usually taken to be 1.06. $\sigma$ is estimated as the
minimum of the usual standard deviation estimator $s$ and
IQR/1.34. This choice of $h$ is known as the normal reference
distribution (nrd) bandwidth.

The following R function was written to compute and plot kernel density
estimates of the cholesterol distribution in each of the behaviour
groups. The Gaussian kernel was used (by default), and smoothing
parameter set to nrd. The output is shown in Fig \ref{fig:kdehist}
superimposed on top of histogram plots.

\begin{Schunk}
\begin{Sinput}
> plot.kde <- function(d) {
+     kde <- density(d,bw="nrd");
+     truehist(d,nbins=6,xlim=c(min(kde$x),max(kde$x)),
+                        ylim=c(min(kde$y),max(kde$y)));
+     lines(kde$x,kde$y);
+ }
\end{Sinput}
\end{Schunk}

\begin{figure}[h]
  \begin{center}
\includegraphics{cw-016}
\includegraphics{cw-017}
\caption{Histogram plots and kernel density estimates of the
  cholesterol distribution in each of the behaviour groups, computed
  using the Gaussian kernel and normal reference distribution
  bandwidth in the \texttt{plot.kde} function.}
\label{fig:kdehist}
\end{center}
\end{figure}

The following code was written to plot the kernel density estimates on
the same chart for comparison.
\begin{Schunk}
\begin{Sinput}
> kde.A <- density(cholA,bw="nrd");
> kde.B <- density(cholB,bw="nrd");
> xlim=c(min(kde.A$x,kde.B$x),max(kde.A$x,kde.B$x));
> ylim=c(min(kde.A$y,kde.B$y),max(kde.A$y,kde.B$y));
> plot(kde.A$x,kde.A$y,type="l",xlim=xlim,ylim=ylim,
+      ylab="probability density",xlab="Cholesterol (mg per 100ml)");
> lines(kde.B$x,kde.B$y,lty=2);
> abline(h=0);
> legend("topright",c("A","B"),lty=1:2);
\end{Sinput}
\end{Schunk}
\includegraphics{cw-018}

The kernel density estimates are far smoother than the histogram
plots, particularly for group B, which looks close to Gaussian save
for the outlier. When plotted side by side, there is a difference in
the modal cholesterol level for the two groups. As seen in the box
plot, cholesterol levels for group B appear to be lower than for group
A. The presence of outliers is likely to influence a test based on
sample means. In particular the location of the outliers would bring
sample means closer together. This effect would be less pronounced on
sample medians.

3. (c) An {\em exact} randomization test is a non-parametric hypothesis test
for the equivalence of treatment groups. The null hypothesis posits
that the treatment group distributions are identical, yet they need
not belong to any specific family. In this test a reference distribution
is obtained by calculating all possible values of the test statistic
under rearrangements of the treatment labels for data points in the
samples. The significance of the observed test statistic is then
calculated with reference to this distribution, as the probability of
incorrectly rejecting the null hypothesis.

It may not be computationally feasibly to carry out every permutation of the
treatment group labels. For instance in the two sample case with equal sample
sizes $n$, there are $\binom{2n}{n}$ permutations. Where an exact
test is not feasible, an {\em approximate} randomization test can be
used. This is a Monte Carlo test where  a (relatively) small sample
from all permutations of the labels is generated randomly to compute the
reference distribution.

In the case of the the behaviour-cholesterol data set, an exact test
of $H_0$ to obtain a $p$-value would require $\binom{40}{20}$
permutations which is of the order $10^{11}$. An approximate test can
be carried out with the following steps:

\begin{enumerate}
  \item Evaluate the statistic $\hat\theta_0$ for the two observed samples
  \item Generate 999 further {\em random} permutations of 20 type A
    and 20 type B labels across the 40 observations, for each permutation
    evaluating and recording $\hat\theta_i$
  \item Since this is a one-way test of $\theta=0$ vs $\theta > 0$,
    $p$ can be estimated as the proportion of all evaluated $\hat\theta$s
    (including $\hat\theta_0$) greater than or equal to $\hat\theta_0$
\end{enumerate}

This test is carried out in the following R code:
\begin{Schunk}
\begin{Sinput}
> ts <- median(cholA)-median(cholB); #calculate observed statistic
> ts;
\end{Sinput}
\begin{Soutput}
[1] 40
\end{Soutput}
\begin{Sinput}
> #set up permutations
> chol <- c(cholA,cholB); #augment samples into one
> n <- length(chol);      #get size of total data
> B <- 999;               #choose number of permutations
> t <- rep(0,B);          #prepare a vector for the statistics
> set.seed(3012);         #set PRNG seed
> #randomly permute the data set B times and record the statistic each time
> for (i in 1:B) {
+  chol.permute <- chol[sample(n)];
+  t[i] <- median(chol.permute[1:(n/2)])- median(chol.permute[(n/2+1):n]);
+ }
> pval <- sum(ts<=c(t,ts))/(B+1); #estimate p-value for the test statistic
> pval;
\end{Sinput}
\begin{Soutput}
[1] 0.018
\end{Soutput}
\end{Schunk}

Clearly $\hat\theta_0>0$ as was observed in the box plots in part
(a). In particular, $\hat\theta_0=40$. The approximate randomization
test estimates the $p$-value of the statistic as 0.018, making it
significant at the $5\%$ but not the $1\%$ level. Therefore we can
conclude from this sample that, at the $5\%$ significance level,
cholesterol level {\em is} related to behaviour type in heavy
middle-aged men: it is higher for men with type A (urgent and
aggressive) behaviour than for men with type B (relaxed, non
competitive) behaviour.

\newpage
4. (a) The conditional distributions $X|N=n$ and $N|X=x$ can be found
by considering the conditioned variable as a constant in the bivariate
joint distribution. This is done for $X|(N=n)$ and $N|(X=x)$ for $x=0$
and $x>0$ separately:
\begin{eqnarray*}
  f(x|N=n) & \propto & \frac{p^x(1-p)^{n-x}}{(n-x)!x!}\\
  X|(N=n) & \sim & Bin(n,p)\\
  \\
  f(n|X=0) & \propto & \frac{(1-p)^n\lambda^n}{n!}\\
  N|(X=0) & \sim & Poi\left((1-p)\lambda\right)\\
  \\
  f(n|X=x) & \propto & \frac{(1-p)^{n-x}\lambda^n}{(n-x)!x!}\\
           & \propto & \frac{\left[(1-p)\lambda\right]^{n-x}}{(n-x)!}\\
  N|(X=x) & = & W + x\\
  \mbox{where } W & \sim & Poi\left((1-p)\lambda\right)\\
  \mbox{and } x & > & 0
\end{eqnarray*}
A caveat to the Poisson distribution found for $N|(X=0)$ is that $n=0$
is part of its support, however the joint pdf is valid for positive $n$
only.

4. (b) The code uses the Gibbs sampling Markov Chain Monte Carlo
algorithm to generate an acyclic, irreducible Markov chain
$\{x_i,n_i\}_{i\ge 1}$ with invariant distribution $f(x,n)$. By the
ergodic asymptotic result, expectations estimated from this chain
converge `in time' to the invariant distribution, including estimates
of the probability mass function (pmf) itself. The Gibbs
sampling algorithm works by sampling from the conditional
distributions $X|N$ and $N|X$ at each step to generate the chain
variable-by-variable, step-by-step. According to the
Metropolis-Hastings acceptance criterion, every sampled observation is
automatically accepted.

The code generates a chain of length 5000, with distributional
parameters $p=0.4$ and $\lambda=20$. \texttt{x1} and \texttt{n1} are
vectors set up to store $\{x_i\}_{i\ge 1}$ and $\{n_i\}_{i\ge
  1}$. These can be used to estimate expectations from the joint or
marginal distributions of $X$ and $Y$ (including their pmfs). The
algorithm implemented is as follows, including a
small modification to ensure the correct support of $f(x,n)$.

\begin{enumerate}
  \item Generate $x_i$ from $X|(N=n_{i-1})$, using $n_0=60$ on the first
    iteration to generate $x_1$, and store it in \texttt{x1[$i$]}:\\
    (*) \texttt{rbinom(1,n,p)}
  \item Generate a candidate $n_i$ from $N|(X=x_i)$, accepting it if it
  is greater than 0, and store it in \texttt{n1[$i$]} (this ensures
  the correct support of $f(x,n)$):\\
  (**)  \texttt{rpois(1,(1-p)*lambda)+x}
  \item Repeat for $i=$ 1 to 5000
\end{enumerate}

4. (c) The completed Gibbs sampling code was run in R, followed by the further code:

\begin{Schunk}
\begin{Sinput}
> #Plot time series and histogram plots of x1 and n1
> plot(x1,type="l",ylab="x",xlab="");
> plot(n1,type="l",ylab="n",xlab="");
> truehist(x1,xlab="x",nbin=20);
> truehist(n1,xlab="n",nbin=20);
> #Remove burn-in periods
> x2 <- x1[501:5000];
> n2 <- n1[501:5000];
> #Calculate summary statistics
> c(mean(x2),var(x2));
\end{Sinput}
\begin{Soutput}
[1] 8.032222 7.818699
\end{Soutput}
\begin{Sinput}
> c(mean(n2),var(n2));
\end{Sinput}
\begin{Soutput}
[1] 20.04022 19.57229
\end{Soutput}
\end{Schunk}

\begin{figure}[h]
\begin{centering}
\includegraphics{cw-022}
\includegraphics{cw-023}
\includegraphics{cw-024}
\caption{Time series and histogram plots of \texttt{x1} and
  \texttt{n1}. Theoretical probability mass functions obtained for $X$
and $N$ are overlaid on the histogram plots.}
\label{fig:Gibbs}
\end{centering}
\end{figure}
The time series and histogram plots are shown in Fig
\ref{fig:Gibbs}. A cursory glance at the time series plots indicates
stationarity of the generated chains, allowing for a burn in period of
perhaps 500 iterations, which is a desirable result. The
histograms, which are approximations to the marginal pmfs of $X$ and $N$,
show two unimodal `bell' shapes although the histogram of \texttt{x1}
hints at asymmetry. The sample mean and variance of the two
chains have also been calculated, allowing for a burn in of 500
iterations.

4. (d) The marginal distributions for $X$ and $N$ are derived from the
joint distribution for $(X,N)$ given in (a):
\begin{eqnarray*}
  f(x) & \propto & \sum_{n=x}^\infty p^x(1-p)^{n-x}\frac{\lambda^n}{(n-x)!x!}\\
       & \propto & \frac{(p\lambda)^x}{x!} \sum_{n=x}^\infty \frac{\left[(1-p)\lambda\right]^{n-x}}{(n-x)!}\\
       & = & \frac{(p\lambda)^x}{x!}\\
  X    & \sim & Poi(p\lambda)\\
  \\
  f(n) & \propto & \sum_{x=0}^n p^x(1-p)^{n-x}\frac{\lambda^n}{(n-x)!x!}\\
       & \propto & \frac{\lambda^n}{n!} \sum_{x=0}^n \dbinom{n}{x}p^x(1-p)^{n-x}\\
       & = & \frac{\lambda^n}{n!}\\
   N   & \sim & Poi(\lambda)
\end{eqnarray*}
The sample means and variances obtained for the two chains
\texttt{x2} and \texttt{n2} in part (c) are consistent with the
theoretical means and variances of these marginal distributions. For
$p=0.4$ and $\lambda=20$, $E[X]=Var(X)=p\lambda=8$ and
$E[N]=Var(N)=\lambda=20$. The pmfs of the two marginal distributions
are plotted over the histograms in Fig \ref{fig:Gibbs} and appear to
be a good match (despite the histograms containing the burn in
period). Further diagnostics could be carried out by carrying out a
test of distributional equality such as a $\chi^2$ test or the
Kolmogorov-Smirnov test.

\end{document}

