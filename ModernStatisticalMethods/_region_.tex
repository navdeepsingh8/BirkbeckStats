\message{ !name(cw.Rnw.tex)}
\message{ !name(cw.Rnw) !offset(15) }

\setkeys{Gin}{width=0.6\textwidth}

1. (a) The cumulative distribution function (cdf) $F(x)$ corresponding to the
Weibull random variable $X$ parameterised by $\rho > 0$ and $\kappa > 0$ is
\begin{equation*}
  F(x) = 1 - e^{-(\rho x)^\kappa}
\end{equation*}
for $x \ge 0$. Since $F(X)$ is uniformly distributed on $[0,1]$ and it
is an increasing function on this interval, we have the result that
for any $U \sim U(0,1)$, $F^{-1}(U)$ has cdf $F$. This result is the
basis for simulation of $X$ via the  inversion method. It is
sufficient to evaluate $F^{-1}(u)$:
\begin{eqnarray*}
  u & = & 1 - e^{-(\rho x)^\kappa} \\
  e^{-(\rho x)^\kappa} & = & 1-u\\
%  e^{-(\rho x)^\kappa} & = & U\\
  (\rho x)^\kappa & = & -\ln(1-u) \\
  x & = & \frac{1}{\rho} \{ -ln(1-u) \} ^{\frac{1}{\kappa}}\\
  \end{eqnarray*}
If $u\sim U(0,1)$ then so is $1-u$. To get the result we replace this with $U$ and $x$ by
$X$ which is a random variable with cdf $F(x)$:
\begin{equation*}
  X = \frac{1}{\rho} \{ -ln(U) \} ^{\frac{1}{\kappa}}
  \end{equation*}

1. (b) A brief summary of the theory behind the acceptance-rejection
(A-R) method:\\
Assume $f(x)$ and $h(x)$ are PDFs such that (the envelope)
$g(x)=kh(x)\ge f(x)$ $\forall x$, $k$ is typically the smallest
positive constant which makes this true
and $h(x)$ has support $[c,d]$. Then for random variables
$X$ and $Y$ if  $Y|(X=x) \sim U(0,g(x))$, $X|(Y\le f(X))$ has PDF
$f(x)$. In the A-R method, $f(x)$ is the PDF we wish to simulate from
and $h(x)$  is a PDF we can simulate from directly.\\
Here $f$ is given by
\begin{equation*}
  f(x) = \frac{1+\alpha}{\alpha(1+x)^2}
\end{equation*}
for $0\le x\le\alpha$ and $\alpha > 0$ and we are instructed to use
the uniform distribution to construct the envelope, i.e.
\begin{equation*}
  h(x) = \frac{1}{\alpha}
\end{equation*}
for $0\le x\le\alpha$. Furthermore we set $k$ to be the smallest
number that ensures $g(x)\ge f(x)$ for $0\le x\le\alpha$. Since $f(x)$
is strictly decreasing in $x$ its maximum value is attained at $x=0$ where
$f(0)=\alpha^{-1}(1+\alpha)$. Therefore we set $k=(1+\alpha)$ giving
the envelope
\begin{equation*}
  g(x) = \frac{1+\alpha}{\alpha}
\end{equation*}
The outline for simulating a sequence $\{x_i\}$ of independent observations from
$f(x)$ is as follows:
\begin{enumerate}
  \item Simulate a candidate observation $x_i \sim U(0,\alpha)$
    i.e. from $h(x)$
  \item Simulate $y_i \sim U(0,\alpha^{-1}(1+\alpha))$ i.e. $U(0,g(x_i))$

  \item If $y_i\le f(x_i)$ accept $x_i$ otherwise reject it
  \item Repeat the steps above
\end{enumerate}

The number of pseudorandom numbers generated per simulated observation
can be evaluated by considering the probability $p$ of accepting a
candidate observation:
\begin{eqnarray*}
 p & = &  \frac{\int_0^\alpha \! f(x) \, dx}{\int_0^\alpha \! g(x) \, dx}\\
   & = &  \frac{\alpha^{-1}(1+\alpha) \int_0^\alpha \! (1+x)^{-2} \, dx}
               {\alpha^{-1}(1+\alpha)\int_0^\alpha \!  \, dx} \\
   & = &  \frac{1}{\alpha} \int_0^\alpha \! (1+x)^{-2} \, dx\\
   & = &  \frac{1}{\alpha}\left[-(1+\alpha)^{-1} \right]^\alpha_0 \\
   & = &  \frac{1}{1+\alpha}
\end{eqnarray*}
Therefore the expected number of candidate observations $x_i$ generated per
simulated observation is $(1+\alpha)$, and since for each $x_i$ a
$y_i$ is also generated, on average $2(1+\alpha)$ pseudorandom numbers
are generated per observation simulated. Clearly, as $\alpha \rightarrow
0$, $f(x)$ tends to the envelope, reducing the number of
pseudorandom numbers required to simulate from it.

\begin{figure}[h]
\begin{center}
<<echo=FALSE,output=FALSE,fig=TRUE>>==
#R code for charts
op <-par(pty="s",mar=c(4.5,4,3,3));
alphas <- c(0.2,0.5,0.8);
x <- list(seq(0,alphas[1],length.out=11),
          seq(0,alphas[2],length.out=11),
          seq(0,alphas[3],length.out=11)
          );
pdf <- function(x) {
    alpha <- max(x);
    (1+alpha)/(alpha*(1+x)^2);
}
y<-lapply(x,pdf);
k<-do.call("rbind",lapply(y,max));
plot(x[[3]],y[[3]],type="l",xlim=c(0,0.8),ylim=c(0,max(k)),xlab=expression(x),ylab="f(x), g(x)");
lines(c(0,alphas[3]),c(k[[3]],k[[3]])); lines(c(alphas[3],alphas[3]),c(0,k[[3]]));
lines(x[[2]],y[[2]],lty=2);
lines(c(0,alphas[2]),c(k[[2]],k[[2]]),lty=2); lines(c(alphas[2],alphas[2]),c(0,k[[2]]),lty=2);
lines(x[[1]],y[[1]],lty=3);
lines(c(0,alphas[1]),c(k[[1]],k[[1]]),lty=3); lines(c(alphas[1],alphas[1]),c(0,k[[1]]),lty=3);
abline(h=0,v=0,lty=5)
legend("topright",as.character(alphas),lty=c(3,2,1),title="alpha");
par(op);
@
\caption{$f(x)$ (curved) and $g(x)$ (horizontal) for $\alpha=$ 0.2,0.5
and 0.8}
\end{center}
\end{figure}


\pagebreak
2. (a) The data were input into R to obtain the data frame \texttt{dugong}, and
plotted.
<<fig=FALSE>>=
dugong=read.table(file(paste(path,"q2.txt",sep="")),header=T);
plot(dugong$age,dugong$length);
@
\begin{figure}[h]
  \begin{center}
<<echo=FALSE,output=FALSE,fig=TRUE>>=
dugong=read.table(file(paste(path,"q2.txt",sep="")),header=T);
plot(dugong$age,dugong$length);
@
\end{center}
\end{figure}
The model is a nonlinear regression model because of the presence of
both $\beta$ and $\gamma$ in the $\beta\gamma^{x_i}$ term. For
strictly non-negative $x_i$ as is the case here, $\mu_i$ is
bounded above by $\alpha$. This aspect of the model is consistent with
the length data, which are bounded above at around 3 metres. Maximum adult dugong
length is an interpretation of $\alpha$. Also, according to the model
$\mu_i$ increases with $x_i$ and is concave, a feature also exhibited in the
data. The extent of the concavity depends on $\gamma$, which can
therefore  be interpreted as a measure of how sensitive growth rate is
to age. $\gamma$ close to 0 corresponds to a much higher growth rate
in youth than in old age. $\gamma$ close to 1 corresponds to a more
stable growth rate through life. Finally $\alpha-\beta$ is the
intercept of the model, the value of $\mu_i$ when $x_i=0$. This can be
interpreted as the size of a newborn dugong, and given $\alpha$ is
governed by $\beta$.

2. (b)
<<fig=FALSE>>=
dugong.fit <- nls(length ~ alpha - beta * gamma^age, dugong,
                  start=list(alpha=2.7,beta=1.0,gamma=0.8));
dugong.fit;
dugong.fit1 <- nls(length ~ alpha - beta * gamma^age, dugong,
                   start=list(alpha=2.5,beta=0.8,gamma=0.6));
dugong.fit1;
plot(dugong$age, dugong$length);
lines(dugong$age, predict(dugong.fit1));
@
\begin{figure}[h]
  \begin{center}
<<echo=FALSE,output=FALSE,fig=TRUE>>=
plot(dugong$age, dugong$length);
lines(dugong$age, predict(dugong.fit1));
@
\end{center}
\end{figure}
The output of the code shows the nonlinear least-square estimates of
$\alpha$,$\beta$ and $\gamma$ starting from two different sets of
starting values. The starting values appear not to be too critical as the
same estimates are arrived at in each case. The plot shows the fitted
values $\hat\mu_i$ against $x_i$, shown as a line (with linear
interpolation between data points). The sum of squared residuals
$(y_i-\hat\mu_i)^2$ has been minimised (via the Gauss-Newton numerical
method) by these parameter estimates. The model appears to
fit well with no (visually) discernable pattern in the
residuals. Further analysis of the residuals could be carried out such
as plots and tests of randomness. We also calculate the ratio of
the model sum-of-squares to the total-sum-of-squares.
<<>>=
total_sum_sq <- (nrow(dugong)-1)*var(dugong$length);
1-deviance(dugong.fit1)/total_sum_sq;
@
$89\%$ of the variance in dugong length is accounted for by the model
which is a good fit.

($\hat\alpha=$) 2.66 metres is the estimated upper bound for dugong
length. ($\hat\alpha$-$\hat\beta$=) 1.69 metres is the estimated
lower bound for dugong length. Also $\hat\gamma=$0.87 corresponds to a
growth rate that is highest before the age of 10, subsequently falling to a
visually lower rate. Estimates of growth rates could be obtained by
linearising parts of the fitted response function and calculating its
gradient.

To carry out classical inference about the precision of these parameter
estimates we would need to make distributional assumptions about the
residuals. Furthermore since the model is nonlinear, we cannot apply
the machinery associated with OLS estimators here.

2. (c) The following modified code was run in R:
<<eval=FALSE,echo=TRUE,output=FALSE>>=
library(boot)
set.seed(965) #Starting at seed 964 causes nls() in R to not converge
dugong.fit <- function(d) {
    coef(nls(length ~ alpha - beta * gamma^age, d,
          start=list(alpha=2.5,beta=0.8,gamma=0.6)));
}
row.fun <- function(d,i) {
 dugong.fit(d[i,]);
}
dugong.boot <- boot(dugong,row.fun,R=1000,stype="i");
@

<<echo=FALSE,output=TRUE>>=
print(dugong.boot);
@

\begin{figure}[h]
  \begin{center}
<<echo=FALSE,outout=FALSE,fig=TRUE,height=7,width=12>>=
plot(dugong.boot,1);
@
<<echo=FALSE,outout=FALSE,fig=TRUE,height=7,width=12>>=
plot(dugong.boot,2);
@
<<echo=FALSE,outout=FALSE,fig=TRUE,height=7,width=12>>=
plot(dugong.boot,3);
@
\caption{The output of the \texttt{plot} function applied to the
  \texttt{dugong.boot} object for each of $\alpha$, $\beta$ and
  $\gamma$ respectively. For each parameter there is a histogram
  of the bootstrap distribution and a $QQ$-plot vs quantiles of the
  standard normal.}
\label{fig:plotboot}
\end{center}
\end{figure}

Empirical bootstrap distributions for the model parameters $\alpha$, $\beta$ and
$\gamma$ (\texttt{t1*},\texttt{t2*} and \texttt{t3*} in the output)
have been computed in this analysis. The idea behind this is
that for a random sample from a population, in the absence of any
other knowledge of that population, the distribution in the sample is
the best guide to the distribution in the population. The population
distribution is approximated by resampling (with replacement) from the
sample. More specifically, the non-linear least squares estimates for
$\alpha$, $\beta$ and $\gamma$ have been calculated for 1000 samples
of the bivariate (age,length) data. Each sample (of equal size to the
original data set) has been obtained by sampling with replacement from
the original data set. The bootstrap distribution for each parameter
is based on these 1000 resampled estimates.

The bias and standard error of each of the parameter estimates have been
estimated from the bootstrap resamples, the bias as follows, using the
bias of $\hat\alpha$, $b(\alpha)=E[\hat\alpha]-\alpha$, as an example:
\begin{equation*}
  \hat{b}(\alpha)= \frac{1}{1000}\sum_{i=1}^{1000}\hat\alpha_i' - \hat\alpha  \\
\end{equation*}
where $\hat\alpha_i'$ is the estimate of $\alpha$ from the $i$th bootstrap
resample and $\hat\alpha$ is the estimate from the original
sample. The rationale for this is that the distribution of
$(\hat\alpha_i'-\hat\alpha)$ mimics that of
$\hat\alpha-\alpha$. $\hat{b}(\alpha)$ and $\hat{b}(\beta)$ are
positive suggesting $\alpha$ and $\beta$ may be slightly overestimated
by $\hat\alpha$ and $\hat\beta$. The amounts in each case are small:
the bias represents about 1.0 and 1.4cm
respectively. $\hat{b}(\gamma)$ is very small and negative.

Applying the \texttt{plot} function to the \texttt{dugong.boot} object
for each parameter gave the output in Fig \ref{fig:plotboot}. For each
parameter there is a histogram of the bootstrap distribution and a
$QQ$-plot vs quantiles of the standard normal.

Estimate 95\% percentile confidence intervals: empirical and BCa (mention assumption)

Overall conclusion from the analysis: inference about the model paramters
fitted in by the nls method

 \end{document}


\message{ !name(cw.Rnw.tex) !offset(-279) }
